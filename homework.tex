
% Default to the notebook output style

    


% Inherit from the specified cell style.




    
\documentclass[11pt]{article}

    
    
    \usepackage[T1]{fontenc}
    % Nicer default font (+ math font) than Computer Modern for most use cases
    \usepackage{mathpazo}

    % Basic figure setup, for now with no caption control since it's done
    % automatically by Pandoc (which extracts ![](path) syntax from Markdown).
    \usepackage{graphicx}
    % We will generate all images so they have a width \maxwidth. This means
    % that they will get their normal width if they fit onto the page, but
    % are scaled down if they would overflow the margins.
    \makeatletter
    \def\maxwidth{\ifdim\Gin@nat@width>\linewidth\linewidth
    \else\Gin@nat@width\fi}
    \makeatother
    \let\Oldincludegraphics\includegraphics
    % Set max figure width to be 80% of text width, for now hardcoded.
    \renewcommand{\includegraphics}[1]{\Oldincludegraphics[width=.8\maxwidth]{#1}}
    % Ensure that by default, figures have no caption (until we provide a
    % proper Figure object with a Caption API and a way to capture that
    % in the conversion process - todo).
    \usepackage{caption}
    \DeclareCaptionLabelFormat{nolabel}{}
    \captionsetup{labelformat=nolabel}

    \usepackage{adjustbox} % Used to constrain images to a maximum size 
    \usepackage{xcolor} % Allow colors to be defined
    \usepackage{enumerate} % Needed for markdown enumerations to work
    \usepackage{geometry} % Used to adjust the document margins
    \usepackage{amsmath} % Equations
    \usepackage{amssymb} % Equations
    \usepackage{textcomp} % defines textquotesingle
    % Hack from http://tex.stackexchange.com/a/47451/13684:
    \AtBeginDocument{%
        \def\PYZsq{\textquotesingle}% Upright quotes in Pygmentized code
    }
    \usepackage{upquote} % Upright quotes for verbatim code
    \usepackage{eurosym} % defines \euro
    \usepackage[mathletters]{ucs} % Extended unicode (utf-8) support
    \usepackage[utf8x]{inputenc} % Allow utf-8 characters in the tex document
    \usepackage{fancyvrb} % verbatim replacement that allows latex
    \usepackage{grffile} % extends the file name processing of package graphics 
                         % to support a larger range 
    % The hyperref package gives us a pdf with properly built
    % internal navigation ('pdf bookmarks' for the table of contents,
    % internal cross-reference links, web links for URLs, etc.)
    \usepackage{hyperref}
    \usepackage{longtable} % longtable support required by pandoc >1.10
    \usepackage{booktabs}  % table support for pandoc > 1.12.2
    \usepackage[inline]{enumitem} % IRkernel/repr support (it uses the enumerate* environment)
    \usepackage[normalem]{ulem} % ulem is needed to support strikethroughs (\sout)
                                % normalem makes italics be italics, not underlines
    \usepackage{mathrsfs}
    

    
    
    % Colors for the hyperref package
    \definecolor{urlcolor}{rgb}{0,.145,.698}
    \definecolor{linkcolor}{rgb}{.71,0.21,0.01}
    \definecolor{citecolor}{rgb}{.12,.54,.11}

    % ANSI colors
    \definecolor{ansi-black}{HTML}{3E424D}
    \definecolor{ansi-black-intense}{HTML}{282C36}
    \definecolor{ansi-red}{HTML}{E75C58}
    \definecolor{ansi-red-intense}{HTML}{B22B31}
    \definecolor{ansi-green}{HTML}{00A250}
    \definecolor{ansi-green-intense}{HTML}{007427}
    \definecolor{ansi-yellow}{HTML}{DDB62B}
    \definecolor{ansi-yellow-intense}{HTML}{B27D12}
    \definecolor{ansi-blue}{HTML}{208FFB}
    \definecolor{ansi-blue-intense}{HTML}{0065CA}
    \definecolor{ansi-magenta}{HTML}{D160C4}
    \definecolor{ansi-magenta-intense}{HTML}{A03196}
    \definecolor{ansi-cyan}{HTML}{60C6C8}
    \definecolor{ansi-cyan-intense}{HTML}{258F8F}
    \definecolor{ansi-white}{HTML}{C5C1B4}
    \definecolor{ansi-white-intense}{HTML}{A1A6B2}
    \definecolor{ansi-default-inverse-fg}{HTML}{FFFFFF}
    \definecolor{ansi-default-inverse-bg}{HTML}{000000}

    % commands and environments needed by pandoc snippets
    % extracted from the output of `pandoc -s`
    \providecommand{\tightlist}{%
      \setlength{\itemsep}{0pt}\setlength{\parskip}{0pt}}
    \DefineVerbatimEnvironment{Highlighting}{Verbatim}{commandchars=\\\{\}}
    % Add ',fontsize=\small' for more characters per line
    \newenvironment{Shaded}{}{}
    \newcommand{\KeywordTok}[1]{\textcolor[rgb]{0.00,0.44,0.13}{\textbf{{#1}}}}
    \newcommand{\DataTypeTok}[1]{\textcolor[rgb]{0.56,0.13,0.00}{{#1}}}
    \newcommand{\DecValTok}[1]{\textcolor[rgb]{0.25,0.63,0.44}{{#1}}}
    \newcommand{\BaseNTok}[1]{\textcolor[rgb]{0.25,0.63,0.44}{{#1}}}
    \newcommand{\FloatTok}[1]{\textcolor[rgb]{0.25,0.63,0.44}{{#1}}}
    \newcommand{\CharTok}[1]{\textcolor[rgb]{0.25,0.44,0.63}{{#1}}}
    \newcommand{\StringTok}[1]{\textcolor[rgb]{0.25,0.44,0.63}{{#1}}}
    \newcommand{\CommentTok}[1]{\textcolor[rgb]{0.38,0.63,0.69}{\textit{{#1}}}}
    \newcommand{\OtherTok}[1]{\textcolor[rgb]{0.00,0.44,0.13}{{#1}}}
    \newcommand{\AlertTok}[1]{\textcolor[rgb]{1.00,0.00,0.00}{\textbf{{#1}}}}
    \newcommand{\FunctionTok}[1]{\textcolor[rgb]{0.02,0.16,0.49}{{#1}}}
    \newcommand{\RegionMarkerTok}[1]{{#1}}
    \newcommand{\ErrorTok}[1]{\textcolor[rgb]{1.00,0.00,0.00}{\textbf{{#1}}}}
    \newcommand{\NormalTok}[1]{{#1}}
    
    % Additional commands for more recent versions of Pandoc
    \newcommand{\ConstantTok}[1]{\textcolor[rgb]{0.53,0.00,0.00}{{#1}}}
    \newcommand{\SpecialCharTok}[1]{\textcolor[rgb]{0.25,0.44,0.63}{{#1}}}
    \newcommand{\VerbatimStringTok}[1]{\textcolor[rgb]{0.25,0.44,0.63}{{#1}}}
    \newcommand{\SpecialStringTok}[1]{\textcolor[rgb]{0.73,0.40,0.53}{{#1}}}
    \newcommand{\ImportTok}[1]{{#1}}
    \newcommand{\DocumentationTok}[1]{\textcolor[rgb]{0.73,0.13,0.13}{\textit{{#1}}}}
    \newcommand{\AnnotationTok}[1]{\textcolor[rgb]{0.38,0.63,0.69}{\textbf{\textit{{#1}}}}}
    \newcommand{\CommentVarTok}[1]{\textcolor[rgb]{0.38,0.63,0.69}{\textbf{\textit{{#1}}}}}
    \newcommand{\VariableTok}[1]{\textcolor[rgb]{0.10,0.09,0.49}{{#1}}}
    \newcommand{\ControlFlowTok}[1]{\textcolor[rgb]{0.00,0.44,0.13}{\textbf{{#1}}}}
    \newcommand{\OperatorTok}[1]{\textcolor[rgb]{0.40,0.40,0.40}{{#1}}}
    \newcommand{\BuiltInTok}[1]{{#1}}
    \newcommand{\ExtensionTok}[1]{{#1}}
    \newcommand{\PreprocessorTok}[1]{\textcolor[rgb]{0.74,0.48,0.00}{{#1}}}
    \newcommand{\AttributeTok}[1]{\textcolor[rgb]{0.49,0.56,0.16}{{#1}}}
    \newcommand{\InformationTok}[1]{\textcolor[rgb]{0.38,0.63,0.69}{\textbf{\textit{{#1}}}}}
    \newcommand{\WarningTok}[1]{\textcolor[rgb]{0.38,0.63,0.69}{\textbf{\textit{{#1}}}}}
    
    
    % Define a nice break command that doesn't care if a line doesn't already
    % exist.
    \def\br{\hspace*{\fill} \\* }
    % Math Jax compatibility definitions
    \def\gt{>}
    \def\lt{<}
    \let\Oldtex\TeX
    \let\Oldlatex\LaTeX
    \renewcommand{\TeX}{\textrm{\Oldtex}}
    \renewcommand{\LaTeX}{\textrm{\Oldlatex}}
    % Document parameters
    % Document title
    \title{BI-PST homework}
    
    
    
    
    

    % Pygments definitions
    
\makeatletter
\def\PY@reset{\let\PY@it=\relax \let\PY@bf=\relax%
    \let\PY@ul=\relax \let\PY@tc=\relax%
    \let\PY@bc=\relax \let\PY@ff=\relax}
\def\PY@tok#1{\csname PY@tok@#1\endcsname}
\def\PY@toks#1+{\ifx\relax#1\empty\else%
    \PY@tok{#1}\expandafter\PY@toks\fi}
\def\PY@do#1{\PY@bc{\PY@tc{\PY@ul{%
    \PY@it{\PY@bf{\PY@ff{#1}}}}}}}
\def\PY#1#2{\PY@reset\PY@toks#1+\relax+\PY@do{#2}}

\expandafter\def\csname PY@tok@vi\endcsname{\def\PY@tc##1{\textcolor[rgb]{0.10,0.09,0.49}{##1}}}
\expandafter\def\csname PY@tok@k\endcsname{\let\PY@bf=\textbf\def\PY@tc##1{\textcolor[rgb]{0.00,0.50,0.00}{##1}}}
\expandafter\def\csname PY@tok@gt\endcsname{\def\PY@tc##1{\textcolor[rgb]{0.00,0.27,0.87}{##1}}}
\expandafter\def\csname PY@tok@nt\endcsname{\let\PY@bf=\textbf\def\PY@tc##1{\textcolor[rgb]{0.00,0.50,0.00}{##1}}}
\expandafter\def\csname PY@tok@sb\endcsname{\def\PY@tc##1{\textcolor[rgb]{0.73,0.13,0.13}{##1}}}
\expandafter\def\csname PY@tok@ow\endcsname{\let\PY@bf=\textbf\def\PY@tc##1{\textcolor[rgb]{0.67,0.13,1.00}{##1}}}
\expandafter\def\csname PY@tok@il\endcsname{\def\PY@tc##1{\textcolor[rgb]{0.40,0.40,0.40}{##1}}}
\expandafter\def\csname PY@tok@cp\endcsname{\def\PY@tc##1{\textcolor[rgb]{0.74,0.48,0.00}{##1}}}
\expandafter\def\csname PY@tok@vg\endcsname{\def\PY@tc##1{\textcolor[rgb]{0.10,0.09,0.49}{##1}}}
\expandafter\def\csname PY@tok@o\endcsname{\def\PY@tc##1{\textcolor[rgb]{0.40,0.40,0.40}{##1}}}
\expandafter\def\csname PY@tok@kt\endcsname{\def\PY@tc##1{\textcolor[rgb]{0.69,0.00,0.25}{##1}}}
\expandafter\def\csname PY@tok@cpf\endcsname{\let\PY@it=\textit\def\PY@tc##1{\textcolor[rgb]{0.25,0.50,0.50}{##1}}}
\expandafter\def\csname PY@tok@se\endcsname{\let\PY@bf=\textbf\def\PY@tc##1{\textcolor[rgb]{0.73,0.40,0.13}{##1}}}
\expandafter\def\csname PY@tok@gu\endcsname{\let\PY@bf=\textbf\def\PY@tc##1{\textcolor[rgb]{0.50,0.00,0.50}{##1}}}
\expandafter\def\csname PY@tok@gh\endcsname{\let\PY@bf=\textbf\def\PY@tc##1{\textcolor[rgb]{0.00,0.00,0.50}{##1}}}
\expandafter\def\csname PY@tok@ni\endcsname{\let\PY@bf=\textbf\def\PY@tc##1{\textcolor[rgb]{0.60,0.60,0.60}{##1}}}
\expandafter\def\csname PY@tok@mi\endcsname{\def\PY@tc##1{\textcolor[rgb]{0.40,0.40,0.40}{##1}}}
\expandafter\def\csname PY@tok@na\endcsname{\def\PY@tc##1{\textcolor[rgb]{0.49,0.56,0.16}{##1}}}
\expandafter\def\csname PY@tok@sh\endcsname{\def\PY@tc##1{\textcolor[rgb]{0.73,0.13,0.13}{##1}}}
\expandafter\def\csname PY@tok@sd\endcsname{\let\PY@it=\textit\def\PY@tc##1{\textcolor[rgb]{0.73,0.13,0.13}{##1}}}
\expandafter\def\csname PY@tok@s1\endcsname{\def\PY@tc##1{\textcolor[rgb]{0.73,0.13,0.13}{##1}}}
\expandafter\def\csname PY@tok@nn\endcsname{\let\PY@bf=\textbf\def\PY@tc##1{\textcolor[rgb]{0.00,0.00,1.00}{##1}}}
\expandafter\def\csname PY@tok@sr\endcsname{\def\PY@tc##1{\textcolor[rgb]{0.73,0.40,0.53}{##1}}}
\expandafter\def\csname PY@tok@ne\endcsname{\let\PY@bf=\textbf\def\PY@tc##1{\textcolor[rgb]{0.82,0.25,0.23}{##1}}}
\expandafter\def\csname PY@tok@c\endcsname{\let\PY@it=\textit\def\PY@tc##1{\textcolor[rgb]{0.25,0.50,0.50}{##1}}}
\expandafter\def\csname PY@tok@mo\endcsname{\def\PY@tc##1{\textcolor[rgb]{0.40,0.40,0.40}{##1}}}
\expandafter\def\csname PY@tok@gp\endcsname{\let\PY@bf=\textbf\def\PY@tc##1{\textcolor[rgb]{0.00,0.00,0.50}{##1}}}
\expandafter\def\csname PY@tok@gi\endcsname{\def\PY@tc##1{\textcolor[rgb]{0.00,0.63,0.00}{##1}}}
\expandafter\def\csname PY@tok@m\endcsname{\def\PY@tc##1{\textcolor[rgb]{0.40,0.40,0.40}{##1}}}
\expandafter\def\csname PY@tok@go\endcsname{\def\PY@tc##1{\textcolor[rgb]{0.53,0.53,0.53}{##1}}}
\expandafter\def\csname PY@tok@kd\endcsname{\let\PY@bf=\textbf\def\PY@tc##1{\textcolor[rgb]{0.00,0.50,0.00}{##1}}}
\expandafter\def\csname PY@tok@gd\endcsname{\def\PY@tc##1{\textcolor[rgb]{0.63,0.00,0.00}{##1}}}
\expandafter\def\csname PY@tok@dl\endcsname{\def\PY@tc##1{\textcolor[rgb]{0.73,0.13,0.13}{##1}}}
\expandafter\def\csname PY@tok@nd\endcsname{\def\PY@tc##1{\textcolor[rgb]{0.67,0.13,1.00}{##1}}}
\expandafter\def\csname PY@tok@kr\endcsname{\let\PY@bf=\textbf\def\PY@tc##1{\textcolor[rgb]{0.00,0.50,0.00}{##1}}}
\expandafter\def\csname PY@tok@fm\endcsname{\def\PY@tc##1{\textcolor[rgb]{0.00,0.00,1.00}{##1}}}
\expandafter\def\csname PY@tok@ge\endcsname{\let\PY@it=\textit}
\expandafter\def\csname PY@tok@kc\endcsname{\let\PY@bf=\textbf\def\PY@tc##1{\textcolor[rgb]{0.00,0.50,0.00}{##1}}}
\expandafter\def\csname PY@tok@s\endcsname{\def\PY@tc##1{\textcolor[rgb]{0.73,0.13,0.13}{##1}}}
\expandafter\def\csname PY@tok@gs\endcsname{\let\PY@bf=\textbf}
\expandafter\def\csname PY@tok@sa\endcsname{\def\PY@tc##1{\textcolor[rgb]{0.73,0.13,0.13}{##1}}}
\expandafter\def\csname PY@tok@nc\endcsname{\let\PY@bf=\textbf\def\PY@tc##1{\textcolor[rgb]{0.00,0.00,1.00}{##1}}}
\expandafter\def\csname PY@tok@w\endcsname{\def\PY@tc##1{\textcolor[rgb]{0.73,0.73,0.73}{##1}}}
\expandafter\def\csname PY@tok@kn\endcsname{\let\PY@bf=\textbf\def\PY@tc##1{\textcolor[rgb]{0.00,0.50,0.00}{##1}}}
\expandafter\def\csname PY@tok@kp\endcsname{\def\PY@tc##1{\textcolor[rgb]{0.00,0.50,0.00}{##1}}}
\expandafter\def\csname PY@tok@cs\endcsname{\let\PY@it=\textit\def\PY@tc##1{\textcolor[rgb]{0.25,0.50,0.50}{##1}}}
\expandafter\def\csname PY@tok@mb\endcsname{\def\PY@tc##1{\textcolor[rgb]{0.40,0.40,0.40}{##1}}}
\expandafter\def\csname PY@tok@sx\endcsname{\def\PY@tc##1{\textcolor[rgb]{0.00,0.50,0.00}{##1}}}
\expandafter\def\csname PY@tok@sc\endcsname{\def\PY@tc##1{\textcolor[rgb]{0.73,0.13,0.13}{##1}}}
\expandafter\def\csname PY@tok@vm\endcsname{\def\PY@tc##1{\textcolor[rgb]{0.10,0.09,0.49}{##1}}}
\expandafter\def\csname PY@tok@ch\endcsname{\let\PY@it=\textit\def\PY@tc##1{\textcolor[rgb]{0.25,0.50,0.50}{##1}}}
\expandafter\def\csname PY@tok@bp\endcsname{\def\PY@tc##1{\textcolor[rgb]{0.00,0.50,0.00}{##1}}}
\expandafter\def\csname PY@tok@nl\endcsname{\def\PY@tc##1{\textcolor[rgb]{0.63,0.63,0.00}{##1}}}
\expandafter\def\csname PY@tok@ss\endcsname{\def\PY@tc##1{\textcolor[rgb]{0.10,0.09,0.49}{##1}}}
\expandafter\def\csname PY@tok@cm\endcsname{\let\PY@it=\textit\def\PY@tc##1{\textcolor[rgb]{0.25,0.50,0.50}{##1}}}
\expandafter\def\csname PY@tok@s2\endcsname{\def\PY@tc##1{\textcolor[rgb]{0.73,0.13,0.13}{##1}}}
\expandafter\def\csname PY@tok@si\endcsname{\let\PY@bf=\textbf\def\PY@tc##1{\textcolor[rgb]{0.73,0.40,0.53}{##1}}}
\expandafter\def\csname PY@tok@nv\endcsname{\def\PY@tc##1{\textcolor[rgb]{0.10,0.09,0.49}{##1}}}
\expandafter\def\csname PY@tok@gr\endcsname{\def\PY@tc##1{\textcolor[rgb]{1.00,0.00,0.00}{##1}}}
\expandafter\def\csname PY@tok@mf\endcsname{\def\PY@tc##1{\textcolor[rgb]{0.40,0.40,0.40}{##1}}}
\expandafter\def\csname PY@tok@no\endcsname{\def\PY@tc##1{\textcolor[rgb]{0.53,0.00,0.00}{##1}}}
\expandafter\def\csname PY@tok@mh\endcsname{\def\PY@tc##1{\textcolor[rgb]{0.40,0.40,0.40}{##1}}}
\expandafter\def\csname PY@tok@err\endcsname{\def\PY@bc##1{\setlength{\fboxsep}{0pt}\fcolorbox[rgb]{1.00,0.00,0.00}{1,1,1}{\strut ##1}}}
\expandafter\def\csname PY@tok@nb\endcsname{\def\PY@tc##1{\textcolor[rgb]{0.00,0.50,0.00}{##1}}}
\expandafter\def\csname PY@tok@nf\endcsname{\def\PY@tc##1{\textcolor[rgb]{0.00,0.00,1.00}{##1}}}
\expandafter\def\csname PY@tok@c1\endcsname{\let\PY@it=\textit\def\PY@tc##1{\textcolor[rgb]{0.25,0.50,0.50}{##1}}}
\expandafter\def\csname PY@tok@vc\endcsname{\def\PY@tc##1{\textcolor[rgb]{0.10,0.09,0.49}{##1}}}

\def\PYZbs{\char`\\}
\def\PYZus{\char`\_}
\def\PYZob{\char`\{}
\def\PYZcb{\char`\}}
\def\PYZca{\char`\^}
\def\PYZam{\char`\&}
\def\PYZlt{\char`\<}
\def\PYZgt{\char`\>}
\def\PYZsh{\char`\#}
\def\PYZpc{\char`\%}
\def\PYZdl{\char`\$}
\def\PYZhy{\char`\-}
\def\PYZsq{\char`\'}
\def\PYZdq{\char`\"}
\def\PYZti{\char`\~}
% for compatibility with earlier versions
\def\PYZat{@}
\def\PYZlb{[}
\def\PYZrb{]}
\makeatother


    % Exact colors from NB
    \definecolor{incolor}{rgb}{0.0, 0.0, 0.5}
    \definecolor{outcolor}{rgb}{0.545, 0.0, 0.0}



    
    % Prevent overflowing lines due to hard-to-break entities
    \sloppy 
    % Setup hyperref package
    \hypersetup{
      breaklinks=true,  % so long urls are correctly broken across lines
      colorlinks=true,
      urlcolor=urlcolor,
      linkcolor=linkcolor,
      citecolor=citecolor,
      }
    % Slightly bigger margins than the latex defaults
    
    \geometry{verbose,tmargin=1in,bmargin=1in,lmargin=1in,rmargin=1in}
    
    

    \begin{document}
    
    
    \maketitle
    
    

    
    \section{case0222 - Cholesterol In Urban And Rural
Guatemalans}\label{case0222---cholesterol-in-urban-and-rural-guatemalans}

\url{https://www.rdocumentation.org/packages/Sleuth2/versions/2.0-4/topics/ex0222}

\subsection{Vypracovali (všichni cvičení st
14:30)}\label{vypracovali-vux161ichni-cviux10denuxed-st-1430}

\begin{itemize}
\tightlist
\item
  Matyáš Skalický (skalimat)
\item
  Martin Vastl (vastlmar)
\item
  Matej Choma (chomamat)
\end{itemize}

\subsection{Popis dat}\label{popis-dat}

Dataset pochází ze studie provedené na guatemalských indiánech. Míra
cholesterolu byla změřena celkem 94 jedincům a byl zaznamenán jejich
původ. Bylo naměřeno 49 pozorování na venkově a 45 ve městě.

\subsection{Formát}\label{formuxe1t}

Dataframe obsahuje 94 pozorování na následujících 2 proměnných:

\begin{itemize}
\tightlist
\item
  \textbf{Cholesterol} - Množství cholesterolu v krvi člověka (v mg/l).
\item
  \textbf{Group} - Proměnná obsahující hodnoty "Rural" a "Urban"
  označující, jestli je subjekt z venkova, nebo z města.
\end{itemize}

\subsection{Zdroj}\label{zdroj}

Ramsey, F.L. and Schafer, D.W. (2002). The Statistical Sleuth: A Course
in Methods of Data Analysis (2nd ed), Duxbury.

    \begin{Verbatim}[commandchars=\\\{\}]
{\color{incolor}In [{\color{incolor}1}]:} \PY{k+kn}{library}\PY{p}{(}Sleuth2\PY{p}{)}
        str\PY{p}{(}ex0222\PY{p}{)}
\end{Verbatim}

    \begin{Verbatim}[commandchars=\\\{\}]
'data.frame':   94 obs. of  2 variables:
 \$ Cholesterol: num  133 134 155 170 175 179 181 184 188 189 {\ldots}
 \$ Group      : Factor w/ 2 levels "Rural","Urban": 2 2 2 2 2 2 2 2 2 2 {\ldots}

    \end{Verbatim}

\section{Úkoly}

    \subsection{Úkol 1}\label{uxfakol-1}
    % \subsubsection{Úkol 1}\label{uxfakol-1}

\begin{quote}
(1b) Načtěte datový soubor a rozdělte sledovanou proměnnou na příslušné
dvě pozorované skupiny. Data stručně popište. Pro každu skupinu zvlášť
odhadněte střední hodnotu, rozptyl a medián příslušného rozdělení.
\end{quote}

    \begin{Verbatim}[commandchars=\\\{\}]
{\color{incolor}In [{\color{incolor}2}]:} rural \PY{o}{\PYZlt{}\PYZhy{}} \PY{k+kp}{subset}\PY{p}{(}ex0222\PY{p}{,} Group\PY{o}{==}\PY{l+s}{\PYZdq{}}\PY{l+s}{Rural\PYZdq{}}\PY{p}{,} Cholesterol\PY{p}{,} drop\PY{o}{=}\PY{k+kc}{TRUE}\PY{p}{)}
        urban \PY{o}{\PYZlt{}\PYZhy{}} \PY{k+kp}{subset}\PY{p}{(}ex0222\PY{p}{,} Group\PY{o}{==}\PY{l+s}{\PYZdq{}}\PY{l+s}{Urban\PYZdq{}}\PY{p}{,} Cholesterol\PY{p}{,} drop\PY{o}{=}\PY{k+kc}{TRUE}\PY{p}{)}
\end{Verbatim}

    \subsubsection{Subjekty z venkova:}
    % \emph{Subjekty z venkova:}

    \begin{Verbatim}[commandchars=\\\{\}]
{\color{incolor}In [{\color{incolor}3}]:} \PY{k+kp}{cat}\PY{p}{(}\PY{l+s}{\PYZdq{}}\PY{l+s}{Rural area indians:\PYZbs{}n\PYZdq{}}\PY{p}{)}
        \PY{k+kp}{cat}\PY{p}{(}\PY{l+s}{\PYZdq{}}\PY{l+s}{ER =\PYZdq{}}\PY{p}{,} \PY{k+kp}{mean}\PY{p}{(}rural\PY{p}{)}\PY{p}{,} \PY{l+s}{\PYZdq{}}\PY{l+s}{\PYZbs{}n\PYZdq{}}\PY{p}{)}
        \PY{k+kp}{cat}\PY{p}{(}\PY{l+s}{\PYZdq{}}\PY{l+s}{varR =\PYZdq{}}\PY{p}{,} var\PY{p}{(}rural\PY{p}{)}\PY{p}{,} \PY{l+s}{\PYZdq{}}\PY{l+s}{\PYZbs{}n\PYZdq{}}\PY{p}{)}
        \PY{k+kp}{cat}\PY{p}{(}\PY{l+s}{\PYZdq{}}\PY{l+s}{median =\PYZdq{}}\PY{p}{,} median\PY{p}{(}rural\PY{p}{)}\PY{p}{)}
\end{Verbatim}

    \begin{Verbatim}[commandchars=\\\{\}]
Rural area indians:
ER = 157 
varR = 1008.458 
median = 152
    \end{Verbatim}

    \subsubsection{Subjekty z města:}
    % \emph{Subjekty z města:}

    \begin{Verbatim}[commandchars=\\\{\}]
{\color{incolor}In [{\color{incolor}4}]:} \PY{k+kp}{cat}\PY{p}{(}\PY{l+s}{\PYZdq{}}\PY{l+s}{Urban area indians:\PYZbs{}n\PYZdq{}}\PY{p}{)}
        \PY{k+kp}{cat}\PY{p}{(}\PY{l+s}{\PYZdq{}}\PY{l+s}{EU =\PYZdq{}}\PY{p}{,} \PY{k+kp}{mean}\PY{p}{(}urban\PY{p}{)}\PY{p}{,} \PY{l+s}{\PYZdq{}}\PY{l+s}{\PYZbs{}n\PYZdq{}}\PY{p}{)}
        \PY{k+kp}{cat}\PY{p}{(}\PY{l+s}{\PYZdq{}}\PY{l+s}{varU =\PYZdq{}}\PY{p}{,} var\PY{p}{(}urban\PY{p}{)}\PY{p}{,} \PY{l+s}{\PYZdq{}}\PY{l+s}{\PYZbs{}n\PYZdq{}}\PY{p}{)}
        \PY{k+kp}{cat}\PY{p}{(}\PY{l+s}{\PYZdq{}}\PY{l+s}{median =\PYZdq{}}\PY{p}{,} median\PY{p}{(}urban\PY{p}{)}\PY{p}{)}
\end{Verbatim}

    \begin{Verbatim}[commandchars=\\\{\}]
Urban area indians:
EU = 216.8667 
varU = 1593.618 
median = 206
    \end{Verbatim}
\clearpage
    \subsection{Úkol 2}\label{ux16fkol-2}

\begin{quote}
(1b) Pro každou skupinu zvlášť odhadněte hustotu a distribuční funkci
pomocí histogramu a empirické distribuční funkce.
\end{quote}

    \begin{Verbatim}[commandchars=\\\{\}]
{\color{incolor}In [{\color{incolor}6}]:} par\PY{p}{(}mfrow \PY{o}{=} \PY{k+kt}{c}\PY{p}{(}\PY{l+m}{2}\PY{p}{,} \PY{l+m}{2}\PY{p}{)}\PY{p}{,}  pty\PY{o}{=}\PY{l+s}{\PYZdq{}}\PY{l+s}{s\PYZdq{}}\PY{p}{)}
        \PY{c+c1}{\PYZsh{} rural}
        hist\PY{p}{(}rural\PY{p}{,} col\PY{o}{=}\PY{l+s}{\PYZdq{}}\PY{l+s}{red\PYZdq{}}\PY{p}{,} main\PY{o}{=}\PY{l+s}{\PYZdq{}}\PY{l+s}{Rural \PYZhy{} histogram\PYZdq{}}\PY{p}{,} probability\PY{o}{=}\PY{n+nb+bp}{T}\PY{p}{,}
            xlab\PY{o}{=}\PY{l+s}{\PYZdq{}}\PY{l+s}{cholesterol mg/l\PYZdq{}}\PY{p}{)}
        plot.ecdf\PY{p}{(}rural\PY{p}{,} col\PY{o}{=}\PY{l+s}{\PYZdq{}}\PY{l+s}{red\PYZdq{}}\PY{p}{,} main\PY{o}{=}\PY{l+s}{\PYZdq{}}\PY{l+s}{Rural \PYZhy{} ecdf\PYZdq{}}\PY{p}{,} xlab\PY{o}{=}\PY{l+s}{\PYZdq{}}\PY{l+s}{cholesterol mg/l\PYZdq{}}\PY{p}{)}
        \PY{c+c1}{\PYZsh{} urban}
        hist\PY{p}{(}urban\PY{p}{,} col\PY{o}{=}\PY{l+s}{\PYZdq{}}\PY{l+s}{blue\PYZdq{}}\PY{p}{,} main \PY{o}{=} \PY{l+s}{\PYZdq{}}\PY{l+s}{Urban \PYZhy{} histogram\PYZdq{}}\PY{p}{,} probability\PY{o}{=}\PY{n+nb+bp}{T}\PY{p}{,} 
            xlab\PY{o}{=}\PY{l+s}{\PYZdq{}}\PY{l+s}{cholesterol mg/l\PYZdq{}}\PY{p}{)}
        plot.ecdf\PY{p}{(}urban\PY{p}{,} col\PY{o}{=}\PY{l+s}{\PYZdq{}}\PY{l+s}{blue\PYZdq{}}\PY{p}{,} main\PY{o}{=}\PY{l+s}{\PYZdq{}}\PY{l+s}{Urban \PYZhy{} ecdf\PYZdq{}}\PY{p}{,} xlab\PY{o}{=}\PY{l+s}{\PYZdq{}}\PY{l+s}{cholesterol mg/l\PYZdq{}}\PY{p}{)}
\end{Verbatim}

    \begin{center}
    \adjustimage{max size={0.9\linewidth}{0.9\paperheight}}{homework_files/homework_9_0.png}
    \end{center}
    { \hspace*{\fill} \\}

\clearpage   
    \subsection{Úkol 3}\label{uxfakol-3}

\begin{quote}
(3b) Pro každou skupinu zvlášť najděte nejbližší rozdělení: Odhadněte
parametry normálního, exponenciálního a rovnoměrného rozdělení. Zaneste
příslušné hustoty s odhadnutými parametry do grafů histogramu.
Diskutujte, které z rozdělení odpovídá pozorovaným datům nejlépe.
\end{quote}

    \paragraph{Odhady rozdělení}\label{odhady-rozdux11blenuxed}\mbox{}\\

Pro provedení odhadu jsou využity funkce \emph{mean()} a \emph{sd()}
zabudované do standardní knihovny jazyka R. Odhad je proveden shodně i
pro množinu urban.

Pro odhad normálního a exponenciálního rozdělení jsme využili momentovou
metodu. Pro odhad uniformního rozdělení metodu maximální věrohodnosti.

\paragraph{Normální rozdělení}\label{normuxe1lnuxed-rozdux11blenuxed}

\begin{Shaded}
\begin{Highlighting}[]
\NormalTok{EX =}\StringTok{ }\KeywordTok{mean}\NormalTok{(rural)}
\NormalTok{s =}\StringTok{ }\KeywordTok{sd}\NormalTok{(rural)}
\end{Highlighting}
\end{Shaded}

\paragraph{Exponenciální
rozdělení}\label{exponenciuxe1lnuxed-rozdux11blenuxed}

\begin{Shaded}
\begin{Highlighting}[]
\NormalTok{lambda =}\StringTok{ }\DecValTok{1}\NormalTok{/}\KeywordTok{mean}\NormalTok{(rural)}
\end{Highlighting}
\end{Shaded}

\paragraph{Uniformní rozdělení}\label{uniformnuxed-rozdux11blenuxed}

\begin{Shaded}
\begin{Highlighting}[]
\NormalTok{a =}\StringTok{ }\KeywordTok{min}\NormalTok{(rural)}
\NormalTok{b =}\StringTok{ }\KeywordTok{max}\NormalTok{(rural)}
\end{Highlighting}
\end{Shaded}
\clearpage 
    \begin{Verbatim}[commandchars=\\\{\}]
{\color{incolor}In [{\color{incolor}10}]:} x \PY{o}{\PYZlt{}\PYZhy{}} \PY{k+kp}{seq}\PY{p}{(}\PY{k+kp}{min}\PY{p}{(}rural\PY{p}{)}\PY{p}{,} \PY{k+kp}{max}\PY{p}{(}rural\PY{p}{)}\PY{p}{,} length\PY{o}{=}\PY{l+m}{100}\PY{p}{)}
         
         \PY{c+c1}{\PYZsh{} hodnoty pro jednotlivá rozložení}
         y\PYZus{}norm \PY{o}{\PYZlt{}\PYZhy{}} dnorm\PY{p}{(}x\PY{p}{,} mean\PY{o}{=}\PY{k+kp}{mean}\PY{p}{(}rural\PY{p}{)}\PY{p}{,} sd\PY{o}{=}sd\PY{p}{(}rural\PY{p}{)}\PY{p}{)}
         y\PYZus{}exp \PY{o}{\PYZlt{}\PYZhy{}} dexp\PY{p}{(}x\PY{p}{,} \PY{l+m}{1}\PY{o}{/}\PY{k+kp}{mean}\PY{p}{(}rural\PY{p}{)}\PY{p}{)}
         y\PYZus{}unif \PY{o}{\PYZlt{}\PYZhy{}} dunif\PY{p}{(}x\PY{p}{,} min\PY{o}{=}\PY{k+kp}{min}\PY{p}{(}rural\PY{p}{)}\PY{p}{,} max\PY{o}{=}\PY{k+kp}{max}\PY{p}{(}rural\PY{p}{)}\PY{p}{)}
         
         hist\PY{p}{(}rural\PY{p}{,} probability\PY{o}{=}\PY{n+nb+bp}{T}\PY{p}{,} main\PY{o}{=}\PY{l+s}{\PYZdq{}}\PY{l+s}{Rural \PYZhy{} odhady rozdeleni\PYZdq{}}\PY{p}{,} xlab\PY{o}{=}\PY{l+s}{\PYZdq{}}\PY{l+s}{cholesterol mg/l\PYZdq{}}\PY{p}{,} 
            ylab\PY{o}{=}\PY{l+s}{\PYZdq{}}\PY{l+s}{hustota\PYZdq{}}\PY{p}{)}
         lines\PY{p}{(}x\PY{p}{,} y\PYZus{}norm\PY{p}{,} col\PY{o}{=}\PY{l+s}{\PYZdq{}}\PY{l+s}{blue\PYZdq{}}\PY{p}{,} lwd\PY{o}{=}\PY{l+m}{2}\PY{p}{)}
         lines\PY{p}{(}x\PY{p}{,} y\PYZus{}exp\PY{p}{,} col\PY{o}{=}\PY{l+s}{\PYZdq{}}\PY{l+s}{violet\PYZdq{}}\PY{p}{,} lwd\PY{o}{=}\PY{l+m}{2}\PY{p}{)} 
         lines\PY{p}{(}x\PY{p}{,} y\PYZus{}unif\PY{p}{,} col\PY{o}{=}\PY{l+s}{\PYZdq{}}\PY{l+s}{green\PYZdq{}}\PY{p}{,} lwd\PY{o}{=}\PY{l+m}{2}\PY{p}{)}
         
         legend\PY{p}{(}\PY{l+s}{\PYZdq{}}\PY{l+s}{topleft\PYZdq{}}\PY{p}{,} inset\PY{o}{=}\PY{l+m}{0.037}\PY{p}{,} fill\PY{o}{=}\PY{k+kt}{c}\PY{p}{(}\PY{l+s}{\PYZdq{}}\PY{l+s}{blue\PYZdq{}}\PY{p}{,}\PY{l+s}{\PYZdq{}}\PY{l+s}{violet\PYZdq{}}\PY{p}{,}\PY{l+s}{\PYZdq{}}\PY{l+s}{green\PYZdq{}}\PY{p}{)}\PY{p}{,} 
                legend\PY{o}{=}\PY{k+kt}{c}\PY{p}{(}\PY{l+s}{\PYZdq{}}\PY{l+s}{normalni rozdeleni\PYZdq{}}\PY{p}{,} \PY{l+s}{\PYZdq{}}\PY{l+s}{exponencialni rozdeleni\PYZdq{}}\PY{p}{,}
                \PY{l+s}{\PYZdq{}}\PY{l+s}{uniformni rozdeleni\PYZdq{}}\PY{p}{)}\PY{p}{)}
\end{Verbatim}

    \begin{center}
    \adjustimage{max size={0.9\linewidth}{0.9\paperheight}}{homework_files/homework_12_0.png}
    \end{center}
    % { \hspace*{\fill} \\}
\clearpage 
    \begin{Verbatim}[commandchars=\\\{\}]
{\color{incolor}In [{\color{incolor}9}]:} x \PY{o}{\PYZlt{}\PYZhy{}} \PY{k+kp}{seq}\PY{p}{(}\PY{k+kp}{min}\PY{p}{(}urban\PY{p}{)}\PY{p}{,} \PY{k+kp}{max}\PY{p}{(}urban\PY{p}{)}\PY{p}{,} length\PY{o}{=}\PY{l+m}{40}\PY{p}{)}
        
        y\PYZus{}norm \PY{o}{\PYZlt{}\PYZhy{}} dnorm\PY{p}{(}x\PY{p}{,} mean\PY{o}{=}\PY{k+kp}{mean}\PY{p}{(}urban\PY{p}{)}\PY{p}{,} sd\PY{o}{=}sd\PY{p}{(}urban\PY{p}{)}\PY{p}{)}
        y\PYZus{}exp \PY{o}{\PYZlt{}\PYZhy{}} dexp\PY{p}{(}x\PY{p}{,} \PY{l+m}{1}\PY{o}{/}\PY{k+kp}{mean}\PY{p}{(}urban\PY{p}{)}\PY{p}{)}
        y\PYZus{}unif \PY{o}{\PYZlt{}\PYZhy{}} dunif\PY{p}{(}x\PY{p}{,} min\PY{o}{=}\PY{k+kp}{min}\PY{p}{(}urban\PY{p}{)}\PY{p}{,} max\PY{o}{=}\PY{k+kp}{max}\PY{p}{(}urban\PY{p}{)}\PY{p}{)}
        
        hist\PY{p}{(}urban\PY{p}{,} probability\PY{o}{=}\PY{n+nb+bp}{T}\PY{p}{,} main\PY{o}{=}\PY{l+s}{\PYZdq{}}\PY{l+s}{Urban \PYZhy{} odhady rozdeleni\PYZdq{}}\PY{p}{,} xlab\PY{o}{=}\PY{l+s}{\PYZdq{}}\PY{l+s}{cholesterol mg/l\PYZdq{}}\PY{p}{,} 
            ylab\PY{o}{=}\PY{l+s}{\PYZdq{}}\PY{l+s}{hustota\PYZdq{}}\PY{p}{)}
        lines\PY{p}{(}x\PY{p}{,} y\PYZus{}norm\PY{p}{,} col\PY{o}{=}\PY{l+s}{\PYZdq{}}\PY{l+s}{blue\PYZdq{}}\PY{p}{,} lwd\PY{o}{=}\PY{l+m}{2}\PY{p}{)}
        lines\PY{p}{(}x\PY{p}{,} y\PYZus{}exp\PY{p}{,} col\PY{o}{=}\PY{l+s}{\PYZdq{}}\PY{l+s}{violet\PYZdq{}}\PY{p}{,} lwd\PY{o}{=}\PY{l+m}{2}\PY{p}{)} 
        lines\PY{p}{(}x\PY{p}{,} y\PYZus{}unif\PY{p}{,} col\PY{o}{=}\PY{l+s}{\PYZdq{}}\PY{l+s}{green\PYZdq{}}\PY{p}{,} lwd\PY{o}{=}\PY{l+m}{2}\PY{p}{)}
        
        legend\PY{p}{(}\PY{l+s}{\PYZdq{}}\PY{l+s}{topleft\PYZdq{}}\PY{p}{,} inset\PY{o}{=}\PY{l+m}{0.037}\PY{p}{,} fill\PY{o}{=}\PY{k+kt}{c}\PY{p}{(}\PY{l+s}{\PYZdq{}}\PY{l+s}{blue\PYZdq{}}\PY{p}{,}\PY{l+s}{\PYZdq{}}\PY{l+s}{violet\PYZdq{}}\PY{p}{,}\PY{l+s}{\PYZdq{}}\PY{l+s}{green\PYZdq{}}\PY{p}{)}\PY{p}{,} 
               legend\PY{o}{=}\PY{k+kt}{c}\PY{p}{(}\PY{l+s}{\PYZdq{}}\PY{l+s}{normalni rozdeleni\PYZdq{}}\PY{p}{,} \PY{l+s}{\PYZdq{}}\PY{l+s}{exponencialni rozdeleni\PYZdq{}}\PY{p}{,}
               \PY{l+s}{\PYZdq{}}\PY{l+s}{uniformni rozdeleni\PYZdq{}}\PY{p}{)}\PY{p}{)}
\end{Verbatim}

    \begin{center}
    \adjustimage{max size={0.9\linewidth}{0.9\paperheight}}{homework_files/homework_13_0.png}
    \end{center}

   \subsection{Úkol 4}\label{uxfakol-4}

\begin{quote}
(1b) Pro každou skupinu zvlášť vygenerujte náhodný výběr o 100 hodnotách
z rozdělení, které jste zvolili jako nejbližší, s parametry odhadnutými
v předchozím bodě. Porovnejte histogram simulovaných hodnot s
pozorovanými daty.
\end{quote}

\noindent Na náš dataset se nejvíc hodí normální rozdělení. Parametry jsme odhadli
pomocí knihovních funkcí \emph{mean()} a \emph{sd()}. Samotný náhodný
výběr jsme vygenerovali následujícím příkazem:

\begin{Shaded}
\begin{Highlighting}[]
\KeywordTok{rnorm}\NormalTok{(}\DecValTok{100}\NormalTok{, }\DataTypeTok{mean=}\KeywordTok{mean}\NormalTok{(urban), }\DataTypeTok{sd=}\KeywordTok{sd}\NormalTok{(urban))}
\end{Highlighting}
\end{Shaded}

    \begin{Verbatim}[commandchars=\\\{\}]
{\color{incolor}In [{\color{incolor}14}]:} y \PY{o}{\PYZlt{}\PYZhy{}} rnorm\PY{p}{(}\PY{l+m}{100}\PY{p}{,} mean\PY{o}{=}\PY{k+kp}{mean}\PY{p}{(}urban\PY{p}{)}\PY{p}{,} sd\PY{o}{=}sd\PY{p}{(}urban\PY{p}{)}\PY{p}{)}
         
    hist\PY{p}{(}urban\PY{p}{,} probability\PY{o}{=}\PY{n+nb+bp}{T}\PY{p}{,} col\PY{o}{=}rgb\PY{p}{(}\PY{l+m}{1}\PY{p}{,} \PY{l+m}{0}\PY{p}{,} \PY{l+m}{0}\PY{p}{,} \PY{l+m}{0.5}\PY{p}{)}\PY{p}{,} ylim\PY{o}{=}\PY{k+kt}{c}\PY{p}{(}\PY{l+m}{0}\PY{p}{,} \PY{l+m}{0.012}\PY{p}{)}\PY{p}{,}
        xlab\PY{o}{=}\PY{l+s}{\PYZdq{}}\PY{l+s}{cholesterol mg/l\PYZdq{}}\PY{p}{,} ylab\PY{o}{=}\PY{l+s}{\PYZdq{}}\PY{l+s}{hustota\PYZdq{}}\PY{p}{,} breaks\PY{o}{=}\PY{l+m}{6}\PY{p}{)}
    hist\PY{p}{(}y\PY{p}{,} probability\PY{o}{=}\PY{n+nb+bp}{T}\PY{p}{,} col\PY{o}{=}rgb\PY{p}{(}\PY{l+m}{0}\PY{p}{,} \PY{l+m}{0}\PY{p}{,} \PY{l+m}{1}\PY{p}{,} \PY{l+m}{0.5}\PY{p}{)}\PY{p}{,} add\PY{o}{=}\PY{n+nb+bp}{T}\PY{p}{,} breaks\PY{o}{=}\PY{l+m}{6}\PY{p}{)}
    legend\PY{p}{(}\PY{l+s}{\PYZdq{}}\PY{l+s}{topleft\PYZdq{}}\PY{p}{,} inset\PY{o}{=}\PY{l+m}{0.037}\PY{p}{,} fill\PY{o}{=}\PY{k+kt}{c}\PY{p}{(}\PY{l+s}{\PYZdq{}}\PY{l+s}{blue\PYZdq{}}\PY{p}{,}\PY{l+s}{\PYZdq{}}\PY{l+s}{red\PYZdq{}}\PY{p}{)}\PY{p}{,} 
            legend\PY{o}{=}\PY{k+kt}{c}\PY{p}{(}\PY{l+s}{\PYZdq{}}\PY{l+s}{namerene hodnoty\PYZdq{}}\PY{p}{,} \PY{l+s}{\PYZdq{}}\PY{l+s}{odhad\PYZdq{}}\PY{p}{)}\PY{p}{)}
\end{Verbatim}

    \begin{center}
    \adjustimage{max size={0.9\linewidth}{0.9\paperheight}}{homework_files/homework_16_0.png}
    \end{center}
    
    \begin{Verbatim}[commandchars=\\\{\}]
{\color{incolor}In [{\color{incolor}20}]:} y \PY{o}{\PYZlt{}\PYZhy{}} rnorm\PY{p}{(}\PY{l+m}{100}\PY{p}{,} mean\PY{o}{=}\PY{k+kp}{mean}\PY{p}{(}rural\PY{p}{)}\PY{p}{,} sd\PY{o}{=}sd\PY{p}{(}rural\PY{p}{)}\PY{p}{)}
         
    hist\PY{p}{(}rural\PY{p}{,} probability\PY{o}{=}\PY{n+nb+bp}{T}\PY{p}{,} col\PY{o}{=}rgb\PY{p}{(}\PY{l+m}{1}\PY{p}{,} \PY{l+m}{0}\PY{p}{,} \PY{l+m}{0}\PY{p}{,} \PY{l+m}{0.5}\PY{p}{)}\PY{p}{,}
        xlab\PY{o}{=}\PY{l+s}{\PYZdq{}}\PY{l+s}{cholesterol mg/l\PYZdq{}}\PY{p}{,} ylab\PY{o}{=}\PY{l+s}{\PYZdq{}}\PY{l+s}{hustota\PYZdq{}}\PY{p}{,} breaks\PY{o}{=}\PY{l+m}{6}\PY{p}{)}
    hist\PY{p}{(}y\PY{p}{,} probability\PY{o}{=}\PY{n+nb+bp}{T}\PY{p}{,} col\PY{o}{=}rgb\PY{p}{(}\PY{l+m}{0}\PY{p}{,} \PY{l+m}{0}\PY{p}{,} \PY{l+m}{1}\PY{p}{,} \PY{l+m}{0.5}\PY{p}{)}\PY{p}{,} add\PY{o}{=}\PY{n+nb+bp}{T}\PY{p}{,} breaks\PY{o}{=}\PY{l+m}{6}\PY{p}{)}
    legend\PY{p}{(}\PY{l+s}{\PYZdq{}}\PY{l+s}{topleft\PYZdq{}}\PY{p}{,} inset\PY{o}{=}\PY{l+m}{0.037}\PY{p}{,} fill\PY{o}{=}\PY{k+kt}{c}\PY{p}{(}\PY{l+s}{\PYZdq{}}\PY{l+s}{blue\PYZdq{}}\PY{p}{,}\PY{l+s}{\PYZdq{}}\PY{l+s}{red\PYZdq{}}\PY{p}{)}\PY{p}{,} 
            legend\PY{o}{=}\PY{k+kt}{c}\PY{p}{(}\PY{l+s}{\PYZdq{}}\PY{l+s}{namerene hodnoty\PYZdq{}}\PY{p}{,} \PY{l+s}{\PYZdq{}}\PY{l+s}{odhad\PYZdq{}}\PY{p}{)}\PY{p}{)}
\end{Verbatim}

    \begin{center}
    \adjustimage{max size={0.9\linewidth}{0.9\paperheight}}{homework_files/homework_17_0.png}
    \end{center}
    { \hspace*{\fill} \\}
\clearpage   
    \subsection{Úkol 5}\label{uxfakol-5}

\begin{quote}
(1b) Pro každou skupinu zvlášť spočítejte oboustranný 95\% konfidenční
interval pro střední hodnotu.
\end{quote}

    \begin{Verbatim}[commandchars=\\\{\}]
{\color{incolor}In [{\color{incolor}5}]:} EU \PY{o}{\PYZlt{}\PYZhy{}} \PY{k+kp}{mean}\PY{p}{(}urban\PY{p}{)}
        s \PY{o}{\PYZlt{}\PYZhy{}} sd\PY{p}{(}urban\PY{p}{)}
        n \PY{o}{\PYZlt{}\PYZhy{}} \PY{k+kp}{length}\PY{p}{(}urban\PY{p}{)}
        error \PY{o}{\PYZlt{}\PYZhy{}} qt\PY{p}{(}\PY{l+m}{0.975}\PY{p}{,} df\PY{o}{=}n\PY{l+m}{\PYZhy{}1}\PY{p}{)}\PY{o}{*}s\PY{o}{/}\PY{k+kp}{sqrt}\PY{p}{(}n\PY{p}{)}
        left \PY{o}{\PYZlt{}\PYZhy{}} EU\PY{o}{\PYZhy{}}error
        right \PY{o}{\PYZlt{}\PYZhy{}} EU\PY{o}{+}error
        
        \PY{k+kp}{cat}\PY{p}{(}\PY{l+s}{\PYZdq{}}\PY{l+s}{oboustranný 95\PYZpc{} konfidenční interval pro střední hodnotu urban:\PYZbs{}n\PYZdq{}}\PY{p}{)}
        \PY{k+kp}{cat}\PY{p}{(}\PY{l+s}{\PYZdq{}}\PY{l+s}{(\PYZdq{}}\PY{p}{,}left\PY{p}{,} \PY{l+s}{\PYZdq{}}\PY{l+s}{, \PYZdq{}}\PY{p}{,} right\PY{p}{,} \PY{l+s}{\PYZdq{}}\PY{l+s}{)\PYZbs{}n\PYZdq{}}\PY{p}{)}
\end{Verbatim}

    \begin{Verbatim}[commandchars=\\\{\}]
oboustranný 95\% konfidenční interval pro střední hodnotu urban:
( 204.8733 ,  228.86 )

    \end{Verbatim}

    \begin{Verbatim}[commandchars=\\\{\}]
{\color{incolor}In [{\color{incolor}6}]:} ER \PY{o}{\PYZlt{}\PYZhy{}} \PY{k+kp}{mean}\PY{p}{(}rural\PY{p}{)}
        s \PY{o}{\PYZlt{}\PYZhy{}} sd\PY{p}{(}rural\PY{p}{)}
        n \PY{o}{\PYZlt{}\PYZhy{}} \PY{k+kp}{length}\PY{p}{(}rural\PY{p}{)}
        error \PY{o}{\PYZlt{}\PYZhy{}} qt\PY{p}{(}\PY{l+m}{0.975}\PY{p}{,} df\PY{o}{=}n\PY{l+m}{\PYZhy{}1}\PY{p}{)}\PY{o}{*}s\PY{o}{/}\PY{k+kp}{sqrt}\PY{p}{(}n\PY{p}{)}
        left \PY{o}{\PYZlt{}\PYZhy{}} ER\PY{o}{\PYZhy{}}error
        right \PY{o}{\PYZlt{}\PYZhy{}} ER\PY{o}{+}error
        
        \PY{k+kp}{cat}\PY{p}{(}\PY{l+s}{\PYZdq{}}\PY{l+s}{oboustranný 95\PYZpc{} konfidenční interval pro střední hodnotu rural:\PYZbs{}n\PYZdq{}}\PY{p}{)}
        \PY{k+kp}{cat}\PY{p}{(}\PY{l+s}{\PYZdq{}}\PY{l+s}{(\PYZdq{}}\PY{p}{,}left\PY{p}{,} \PY{l+s}{\PYZdq{}}\PY{l+s}{, \PYZdq{}}\PY{p}{,} right\PY{p}{,} \PY{l+s}{\PYZdq{}}\PY{l+s}{)\PYZbs{}n\PYZdq{}}\PY{p}{)}
\end{Verbatim}

    \begin{Verbatim}[commandchars=\\\{\}]
oboustranný 95\% konfidenční interval pro střední hodnotu rural:
( 147.8785 ,  166.1215 )

    \end{Verbatim}

\clearpage
    \subsection{Úkol 6}\label{uxfakol-6}

\begin{quote}
(1b) Pro každou skupinu zvlášť otestujte na hladině významnosti 5\%
hypotézu, zda je střední hodnota rovná hodnotě K (parametr úlohy), proti
oboustranné alternativě. Můžete použít buď výsledek z předešlého bodu,
nebo výstup z příslušné vestavěné funkce vašeho softwaru.
\end{quote}

    \begin{Verbatim}[commandchars=\\\{\}]
{\color{incolor}In [{\color{incolor}23}]:} alternative \PY{o}{\PYZlt{}\PYZhy{}} \PY{l+s}{\PYZdq{}}\PY{l+s}{two.sided\PYZdq{}}
         K\PYZus{}parameter \PY{o}{\PYZlt{}\PYZhy{}} \PY{l+m}{2}
         conf\PYZus{}level \PY{o}{\PYZlt{}\PYZhy{}} \PY{l+m}{0.95}
         t.test\PY{p}{(}urban\PY{p}{,} mu\PY{o}{=}K\PYZus{}parameter\PY{p}{,} alternative\PY{o}{=}alternative\PY{p}{,} conf.level \PY{o}{=} conf\PYZus{}level\PY{p}{)}
         t.test\PY{p}{(}rural\PY{p}{,} mu\PY{o}{=}K\PYZus{}parameter\PY{p}{,} alternative\PY{o}{=}alternative\PY{p}{,} conf.level \PY{o}{=} conf\PYZus{}level\PY{p}{)}
\end{Verbatim}

    
    \begin{verbatim}

    One Sample t-test

data:  urban
t = 36.106, df = 44, p-value < 2.2e-16
alternative hypothesis: true mean is not equal to 2
95 percent confidence interval:
 204.8733 228.8600
sample estimates:
mean of x 
 216.8667 

    \end{verbatim}

    
    
    \begin{verbatim}

    One Sample t-test

data:  rural
t = 34.167, df = 48, p-value < 2.2e-16
alternative hypothesis: true mean is not equal to 2
95 percent confidence interval:
 147.8785 166.1215
sample estimates:
mean of x 
      157 

    \end{verbatim}

    
\paragraph{Rural:}Testovaná hodnota $ER=K=2$ v intervalu neleží, takže můžeme
hypotézu vyváženosti na hladině významnosti 5\% zamítnout ve prospěch
alternativy, že je pravděpodobnost, že $ER=2$ je významně odlišná.

\paragraph{Urban:}Testovaná hodnota $EU=K=2$ v intervalu neleží, takže můžeme
hypotézu vyváženosti na hladině významnosti 5\% zamítnout ve prospěch
alternativy, že je pravděpodobnost, že $EU=2$ je významně odlišná.

\clearpage
    \subsection{Úkol 7}\label{uxfakol-7}

\begin{quote}
(2b) Na hladině spolehlivosti 5\% otestujte, jestli mají pozorované
skupiny stejnou střední hodnotu. Typ testu a alternativy stanovte tak,
aby vaše volba nejlépe korespondovala s povahou zkoumaného problému.
\end{quote}

\noindent Z povahy testovaných skupin můžeme předpokládat, že střední hodnota
cholesterolu u indiánů žijících ve městě bude vyšší, něž u těch žijících
na venkově. Proto vybíráme alternativní hypotézu $H_A:\ EU > ER$

    \begin{Verbatim}[commandchars=\\\{\}]
{\color{incolor}In [{\color{incolor}6}]:} EU \PY{o}{\PYZlt{}\PYZhy{}} \PY{k+kp}{mean}\PY{p}{(}urban\PY{p}{)}
        ER \PY{o}{\PYZlt{}\PYZhy{}} \PY{k+kp}{mean}\PY{p}{(}rural\PY{p}{)}
        \PY{k+kp}{cat}\PY{p}{(}\PY{l+s}{\PYZdq{}}\PY{l+s}{Střední hodnota pro urban EU =\PYZdq{}}\PY{p}{,}EU\PY{p}{,}\PY{l+s}{\PYZdq{}}\PY{l+s}{\PYZbs{}n\PYZdq{}}\PY{p}{)}
        \PY{k+kp}{cat}\PY{p}{(}\PY{l+s}{\PYZdq{}}\PY{l+s}{Střední hodnota pro rural ER =\PYZdq{}}\PY{p}{,}ER\PY{p}{,}\PY{l+s}{\PYZdq{}}\PY{l+s}{\PYZbs{}n\PYZdq{}}\PY{p}{)}
        
        alternative \PY{o}{\PYZlt{}\PYZhy{}} \PY{l+s}{\PYZdq{}}\PY{l+s}{greater\PYZdq{}}
        conf\PYZus{}level \PY{o}{\PYZlt{}\PYZhy{}} \PY{l+m}{0.95}
        t.test\PY{p}{(}urban\PY{p}{,} mu\PY{o}{=}ER\PY{p}{,} alternative\PY{o}{=}alternative\PY{p}{,} conf.level \PY{o}{=} conf\PYZus{}level\PY{p}{)}
\end{Verbatim}

    \begin{Verbatim}[commandchars=\\\{\}]
Střední hodnota pro urban EU = 216.8667 
Střední hodnota pro rural ER = 157 
    \end{Verbatim}

    
    \begin{verbatim}

    One Sample t-test

data:  urban
t = 10.06, df = 44, p-value = 2.777e-13
alternative hypothesis: true mean is greater than 157
95 percent confidence interval:
 206.8677      Inf
sample estimates:
mean of x 
 216.8667 
    \end{verbatim}

\noindent Testujeme hypotézu $H_0:\ EU = ER$ proti alternativě $H_A:\ EU > ER$. 
Jednostranný $95\%$ konfidenční interval pro $EU$ je $(206.8677,+\infty)$. 
Střední hodnota $ER = 157$ v intervalu neleží. 
Hypotézu $H_0$ na hladině významnosti $5\%$ můžeme tedy zamítnout ve prospěch alternativy $H_A$.

    % Add a bibliography block to the postdoc
    
    
    
    \end{document}
